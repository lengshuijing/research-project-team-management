\section{Introduction}
\label{sec:Introduction}

Large Language Models (LLMs) have shown remarkable capabilities in various domains,
including natural language understanding, text generation, and problem-solving. Recent
research has explored the potential of using LLMs as agents that can interact with
each other and their environment to perform complex tasks \cite{li2023camelcommunicativeagentsmind,yao2023reactsynergizingreasoningacting,wang_2024,schick2023toolformerlanguagemodelsteach}.
However, existing LLM agents face limitations such as hallucination, where they
generate incorrect or nonsensical information, and restricted context windows,
limiting their ability to handle long documents or conversations
\cite{hao2023reasoninglanguagemodelplanning}. Additionally, coordinating
multiple LLM agents effectively remains a challenge. Existing approaches often lack
clear hierarchical structures and communication protocols, leading to
inefficiencies and potential conflicts, as highlighted in \cite{feng2024agilenovelreinforcementlearning}
which mentions the absence of comprehensive benchmarks addressing these challenges
in real-world agent applications.

This paper introduces the Chain-of-Command Framework (CoCF) to address these challenges.
Inspired by military hierarchies, CoCF establishes a clear chain of command among
LLM agents, with different agents assigned specific roles and responsibilities. This
hierarchical structure facilitates efficient task delegation, feedback, and coordination,
mitigating the limitations of individual agents by enabling collective reasoning
and specialized task execution. The CoCF language provides a formal way for agents
to communicate and interact within the hierarchy, ensuring clear and unambiguous
communication.

The paper is organized as follows: \textit{Analysis} analyzes key concepts and
ideas relevant to the context by drawing connections between different
approaches to inform the design of CoCF. \textit{Design} details the design of
the CoCF language, including its syntax, semantics, and communication protocols,
incorporating principles from the AGILE and MOA frameworks. \textit{Implementation}
describes the implementation of the CoCF framework and the prototype system. \textit{Experimental
Design} presents the experimental setup and discusses evaluation metrics, including
the Pareto frontier and cost-aware comparisons. Finally, \textit{Discussion and
Future Directions} and \textit{Conclusion} discuss potential applications,
challenges, and future research directions.

\section{Analysis}
\label{sec:Analysis}

Large language models (LLMs) can act as powerful problem-solvers yet remain
prone to issues such as hallucinations and limited context handling
\cite{li2023camelcommunicativeagentsmind,yao2023reactsynergizingreasoningacting,wang_2024,schick2023toolformerlanguagemodelsteach}.
Researchers have noted that multi-agent configurations help mitigate these drawbacks
by enabling agents to verify one another's outputs and distribute cognitive load,
thereby improving robustness and efficiency
\cite{li2024agentsneed,wang2024mixtureofagentsenhanceslargelanguage,wu2024perhapshumantranslationharnessing,cheng2025selfplayingadversariallanguagegame}.
Meanwhile, the notion of hierarchical organization--drawn from both AI and human
team structures--suggests that a clear chain of command, when supported by well-defined
communication protocols like KQML or FIPA-ACL, can streamline information
exchange and align agents toward shared objectives
\cite{tao2024magisllmbasedmultiagentframework}. Researchers further propose evolutionary
algorithms as a means of optimizing these multi-agent systems, demonstrating how
iterative processes can fine-tune factors such as prompt strategies and agent
configurations \cite{akiba_2025}. Beyond these technical mechanisms, studies on
frameworks such as AGILE emphasize the value of planning, reflection, and
experiential learning for enhancing adaptability in LLM-driven systems
\cite{feng2024agilenovelreinforcementlearning}, while the Mixture of Agents (MOA)
approach illustrates how combining diverse LLMs can produce synergistic gains \cite{wang2024mixtureofagentsenhanceslargelanguage}.
These findings not only highlight the complexity inherent in orchestrating
multiple intelligent agents but also illuminate the promising avenues for
designing CoCF, from reinforcing hierarchical structures and communication protocols
to harnessing evolutionary optimization and multi-agent collaboration for more dynamic,
scalable outcomes.

\section{Design}
\label{sec:Design}

\subsection{Design of CoCF Language}

The CoCF language is designed to be human-readable and easily interpretable by
LLMs. It uses a simple syntax based on keywords and natural language instructions,
enabling seamless communication and coordination within the hierarchical agent
structure.

\subsection{Syntax and Semantics}

The basic structure of CoCF is as follows:

\begin{verbatim}
  <Rank> <Agent ID>: <Command> <Task Description> <Parameters>
\end{verbatim}

The \textit{Rank} indicates the hierarchical level of the agent issuing the command
(e.g., General, Officer, Soldier), establishing the authority and responsibility
for task delegation and feedback. The \textit{Agent ID} is a unique identifier
for the agent, ensuring clear identification and addressing within the hierarchy.
The \textit{Command} specifies the action to be performed (e.g., \textit{EXECUTE},
\textit{REPORT}, \textit{REQUEST}), defining the core communication protocols for
task management and coordination. The \textit{Task Description} provides a natural
language description of the task, allowing for flexible and expressive task
assignments. Finally, \textit{Parameters} are optional parameters for the task,
providing further specificity and control over task execution. For example, a
General agent might issue the following command to an Officer agent:

\begin{verbatim}
  General G1: EXECUTE "Analyze market trends for electric vehicles"
  { timeframe: "next 5 years", region: "North America" }
\end{verbatim}

This command instructs the Officer agent with ID \textit{O2} to perform the
specified market analysis with the given parameters.

CoCF defines a hierarchical structure with different LLM agents assigned
specific roles and responsibilities, inspired by military ranks. The General is
the highest-ranking agent in the CoCF hierarchy, responsible for overall strategy
and task delegation. Generals receive high-level objectives and decompose them
into smaller sub-tasks, assigning these sub-tasks to Officer agents. They also
oversee the progress of the entire operation, receive reports from Officers, and
make adjustments to the strategy as needed. Officers are middle-ranking agents responsible
for overseeing the execution of sub-tasks assigned by Generals. Officers further
decompose these sub-tasks into even smaller tasks and delegate them to Soldier
agents. They monitor the progress of Soldier agents, provide guidance and
support, and aggregate the results to report back to the General. Soldiers are
the lowest-ranking agents responsible for executing individual tasks assigned by
Officers. Soldiers may use tools, access external resources, or interact with the
environment to complete their assigned tasks. They report their progress, results,
and any encountered issues back to their supervising Officer. This hierarchical structure,
inspired by military organizations, ensures clear lines of authority and responsibility,
facilitating efficient task management and coordination. Each agent has a well-defined
role and understands its responsibilities within the hierarchy, minimizing
confusion and conflicts.

CoCF establishes clear communication protocols for task delegation, feedback,
and coordination, drawing inspiration from established agent communication languages
like KQML and FIPA-ACL. Higher-ranking agents delegate tasks to lower-ranking
agents using \textit{EXECUTE} commands, which include a clear task description
and any necessary parameters. Lower-ranking agents provide feedback to higher-ranking
agents using \textit{REPORT} commands, including progress updates, results of
completed tasks, and any potential issues encountered. Agents can use \textit{REQUEST}
commands to seek information or assistance from other agents within the hierarchy,
facilitating collaboration and information sharing between agents. To further standardize
communication, CoCF utilizes the concept of ``protocol documents (PDs)''
introduced in \cite{marro2024scalablecommunicationprotocolnetworks}. PDs provide
a machine-readable description of the communication protocols, ensuring that all
agents have a clear and consistent understanding of the communication rules. This
reduces ambiguity and facilitates interoperability between different agents.

CoCF incorporates elements from the AGILE and MOA frameworks to enhance its capabilities.
It integrates the AGILE framework's principles of planning, reflection, and
learning \cite{feng2024agilenovelreinforcementlearning}. Agents are encouraged to
plan their tasks before execution, breaking down complex tasks into smaller, more
manageable steps. After completing a task, agents reflect on their performance,
identifying areas for improvement. This reflection process facilitates learning from
experience, enabling agents to adapt and improve their strategies over time. CoCF
also adopts the MOA framework's approach of combining multiple LLMs to enhance performance
\cite{wang2024mixtureofagentsenhanceslargelanguage}. This can be achieved by
having multiple Soldier agents work on the same task independently. Their individual
outputs are then aggregated by an Officer agent, which synthesizes the different
perspectives and generates a final result. This approach leverages the strengths
of different LLMs and reduces the risk of individual biases or errors.

\section{Implementation}
\label{sec:Implementation}

The CoCF language is implemented using Python and the OpenAI API. A prototype system
is developed with different LLM agents interacting within the CoCF hierarchy.
The system includes separate classes for General, Officer, and Soldier agents, each
with specific methods for communication, task execution, and feedback. These
classes encapsulate the roles and responsibilities of each agent type within the
CoCF hierarchy. A communication module handles message passing between agents according
to the CoCF communication protocols, ensuring that messages are correctly
formatted, routed to the appropriate recipients, and processed according to the
defined protocols. A task execution module manages the execution of tasks by Soldier
agents, providing mechanisms for agents to access and use tools, interact with external
resources, and perform computations. A feedback and reporting module processes feedback
from lower-ranking agents and generates reports for higher-ranking agents. It
aggregates results, summarizes progress, and identifies potential issues to
provide a comprehensive overview of the task execution. The prototype system also
utilizes TextGrad, to ``differentiate'' text and prioritize novel-seeking
behaviors.

\section{Experiment Design}
\label{sec:Experiment}

To evaluate the performance and effectiveness of CoCF, experiments are designed
to compare its performance with single LLM agents and other multi-agent
frameworks. The experiments focus on tasks that require coordination, planning, and
problem-solving, such as collaborative writing, complex question answering, and
strategic decision making. The experiments can utilize established benchmarks and
datasets relevant to each task, such as the HotpotQA dataset for complex
question answering and the MultiWOZ dataset for strategic decision making.

The results of the experiments are analyzed based on various metrics, including task
completion rate, accuracy, efficiency, and communication overhead. In addition to
these metrics, the evaluation incorporates the concept of ``Pareto frontier''
from \cite{kapoor2024aiagentsmatter} to analyze the trade-off between different
performance metrics. For example, we can evaluate the trade-off between accuracy
and efficiency, identifying CoCF configurations that achieve the best balance between
these two objectives. The evaluation also considers cost-aware comparisons, as highlighted
in \cite{kapoor2024aiagentsmatter}, taking into account the computational cost
of using different LLMs and the overall cost of running the CoCF framework.

\section{Discussion and Future Directions}
\label{sec:discussion}

The CoCF language and framework offer a promising approach for coordinating LLM agents
in complex tasks. The hierarchical structure and clear communication protocols
facilitate efficient task delegation, feedback, and coordination. By enabling specialized
agents to collaborate and reason collectively, CoCF can mitigate limitations of
current LLM agents, such as hallucination and limited context windows.

CoCF has the potential to be applied in various domains, including software
development, content creation, and research and analysis. Challenges like scalability,
adaptability, robustness, and explainability will be addressed through further
research and development. The potential for agent protocol interoperability will
be investigated to enable CoCF agents to interact with agents from other frameworks.
Potential drawbacks, such as increased communication overhead and potential bottlenecks,
will be analyzed and mitigated through strategies like distributed decision-making
or redundant agents.

CoCF, as a framework, utilizes the principle of ``inversion of control''. This means
that the framework itself dictates the flow of control, calling into the agent
code as needed. This approach provides a standardized structure and ensures that
agents adhere to the defined communication protocols and task execution procedures.

By addressing these challenges and exploring future research directions, CoCF
can become a valuable tool for developing and deploying LLM agents in various
real-world applications.

\section{Conclusion}
\label{sec:conclusion}

This paper introduced the Chain-of-Command Framework (CoCF), a novel language
and framework for coordinating LLM agents. CoCF leverages a hierarchical structure
inspired by military organizations and clear communication protocols to improve the
efficiency and effectiveness of LLM agents in complex tasks. By incorporating
principles from the AGILE and MOA frameworks, CoCF enables agents to plan,
reflect, learn, and collaborate effectively. The paper detailed the design and implementation
of the CoCF language, presented a prototype system, and discussed potential applications,
challenges, and future research directions.

In conclusion, CoCF offers several key advantages: improved coordination,
reduced hallucination, enhanced context handling, increased adaptability, and leveraging
multiple LLMs. CoCF has the potential to significantly impact various domains, including
software development, content creation, and research and analysis.