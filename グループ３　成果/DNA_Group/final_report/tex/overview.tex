
\section{Overview}
\label{sec:overview}
\section*{Background and Overview}

Artificial intelligence (AI) has made remarkable strides in recent years, with large language models (LLMs) such as OpenAI’s GPT-4 achieving capabilities comparable to, and sometimes surpassing, human performance in various domains. These models have demonstrated proficiency in natural language processing (NLP), reasoning, and complex problem-solving tasks. However, while GPT-4 and similar LLMs exhibit remarkable individual performance, they remain limited as single-agent systems, often struggling with inconsistencies, contextual misunderstandings, and generating highly domain-specific insights.

Interestingly, humans, the dominant species on Earth, provide a compelling example of the power of collective intelligence. \cite{mountcastle1997columnar} Human societies excel not only due to individual abilities but because of their capacity to collaborate, leveraging diverse skills and perspectives to solve challenges. At the same time, each human being can be seen as a multi-agent system, where the brain's cortical regions—organized into specialized areas analogous to agents—coordinate seamlessly to produce unified, intelligent behavior. Studies suggest that cortical circuits exhibit a level of specialization and communication that rivals the capabilities of large AI systems, forming an integrated and adaptive decision-making unit. \cite{mountcastle1997columnar}\cite{tononi1998consciousness}

Building on this inspiration, this research aims to enhance the collective intelligence of AI systems by designing multi-agent frameworks that emulate human-like collaborative structures. Specifically, we develop a framework for assessing and improving the collective capabilities of multi-agent systems. Unlike approaches that modify the internal mechanisms of individual agents, our method optimizes the social topology and communication protocols of the collective system using evolutionary algorithms. This allows the system to evolve and adapt, forming more effective collaborative structures without altering the pretrained agents themselves.

To evaluate the system, we will introduce a performance metric based on spectral analysis of standard benchmark scores. This metric provides a generalizable way to assess the emergent intelligence of the collective system, offering insights into its capabilities across diverse tasks.

In this research, we implemented and compared two prominent multi-agent frameworks:
\begin{enumerate}
  \item \textbf{OpenAI Swarm Framework with GPT-4:} A foundational platform for orchestrating agent interactions and role allocation. \cite{openai}
  \item \textbf{Microsoft Autogen Framework with LLaMA 3.2 (3B):} A more flexible and scalable system integrated with Ollama, offering advanced features like local processing without the need of using API. \cite{microsoft} \cite{autogen_ollama_documentation}
\end{enumerate}

By combining the strengths of swarm intelligence and multi-agent collaboration, this work seeks to address the intrinsic limitations of standalone LLMs and move toward robust, decentralized AI systems. Just as humans as a species have leveraged their collective intelligence to overcome complex challenges, this research aims to unlock the potential of AI as a collective system, paving the way for scalable, adaptive, and contextually aware AI frameworks.
