\title{Collective Intelligence - DNA Model}

\author{
    \bf Ye Yuhe (Jason) \\
    \texttt{jason.ye@keio.jp} \\
    \And
    \bf Gu Yu \\
    \texttt{ygu.cju@gmail.com}
    \And
    \bf Federico Peitti \\
    \texttt{fpeitti@keio.jp}
    \And
    \bf Yu Yueyang (Sakana) \\
    \texttt{yakuyou.jp@keio.jp}
}

\date{}
\maketitle
\onehalfspacing

\begin{abstract}
This report explores the development of a Collective Intelligence DNA Model, an AI framework inspired by human cognition and collaborative intelligence. While large language models (LLMs) such as OpenAI’s GPT-4 and Microsoft’s LLaMA exhibit remarkable problem-solving capabilities, they are inherently limited as standalone entities. Human intelligence, by contrast, thrives through specialization, hierarchy, and structured decision-making. This research seeks to emulate these principles by introducing a hierarchically structured multi-agent system that improves AI collaboration and problem-solving efficiency.

At the core of the model is the DNA node, responsible for task allocation and iterative evaluation. Supporting it are stem cell nodes, which represent specialized agents collaborating to refine responses through discussion, and an aggregator node, which consolidates diverse insights. A key inspiration for this model is hierarchical organization systems like military rank structures, which efficiently distribute authority and decision-making responsibilities. To explore this concept, this research integrates an Army Hierarchy framework, a structured ranking-based system that mirrors the DNA Model’s decision-making architecture. By mapping AI agents onto a military-style chain of command, this approach optimizes information flow, task delegation, and problem-solving efficiency.
\end{abstract}
